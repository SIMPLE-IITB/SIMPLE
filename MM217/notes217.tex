\documentclass[12pt]{article}
\usepackage{xparse}
\usepackage[utf8]{inputenc}
\usepackage{graphicx}
\usepackage{amsmath}
\usepackage{calligra}
\usepackage{asymptote}
\usepackage[version=4]{mhchem}
\usepackage{titlesec}
\setcounter{secnumdepth}{4}



\usepackage{enumitem}

\newcommand{\scriptr}{\mathcalligra{r}\,}
\newcommand{\boldscriptr}{\pmb{\mathcalligra{r}}\,}
\usepackage{physics}
\usepackage{tcolorbox}
\usepackage{tikz}
\usepackage[a4paper]{geometry}
\usepackage{hyperref}
\title{MM 217 Formulae}
\author{{\color{blue}Advait Risbud}}
\date{May 21'}

\begin{document}
	\maketitle
	\tableofcontents
	\pagebreak
	\section{Descriptive Statistics}
	\subsection{Measures of central tendency}
	\paragraph*{Mean} If $ x_1, x_2, x_3,\cdots x_n $ are data points then the mean is given by
	\begin{equation}
		\bar{x}=\dfrac{\sum_{i=1}^{n}x_i}{n}
	\end{equation}
	R Command: {\ttfamily mean(data)}
	\paragraph*{Median} It is the value which divides the data in half. Let n denote the number of data points then if 
	\begin{itemize}
		\item n=even, then 
		\begin{equation}
			\tilde{x}=x_{\frac{n}{2}}+x_{\frac{n+2}{2}}
		\end{equation}
	
	    \item n=odd, the
	    \begin{equation}
	    	\tilde{x}=x_{\frac{n+1}{2}}
	    \end{equation}
	R command: {\ttfamily median(data)}
	\end{itemize}
	\paragraph*{Mode} Data point with the highest frequency. \\
	R command: {\ttfamily mode(data)}
	
	\paragraph*{Variance}
	Variance is $ \sigma^2 $ and standard deviation is $\sigma$.
	\begin{equation}
		\sigma^2=\dfrac{\sum_{i=1}^{n}(x_i-\bar{x})^2}{n-1}
	\end{equation}
	Between one sigma limits of a normal distribution 67\% of the data is contained. \\
	R command: {\ttfamily sd(data)}
	\subsection{Graphical Methods}
	Empty for now, will add post midsems.
	\subsection{Correlation Coefficient}
	Consider two data sets $ \{x_1,\cdots,x_n \}$ and $ \{y_1,\cdots, y_n \}$. Covariance and correlation coefficient are defined as,
	\begin{align}
		Cov(x,y)&=\sum_{i=1}^{n}(x_i-\bar{x})(y_i-\bar{y})\times \frac{1}{n-1}\\
		Corr(x,y)&=\dfrac{Cov(x,y)}{\sqrt{Var(x)Var(y)}}
	\end{align}
	\begin{itemize}
		\item $ -1\leq Corr(x,y)\leq 1 $
		\item $Corr(x,y)=1  \implies $ perfect linear relationship with positive slope.
		\item $Corr(x,y)=-1  \implies $ perfect linear relationship with negative slope.
		\item Intermediate values indicate different extent of linearity in the two variables.
	\end{itemize}
R command: {\ttfamily cov(x,y)} and {\ttfamily cor(x,y)}.

    \section{Probability}



	
	
	
	
	
	
	
	
	
	
	
	
	
	
	
	
\end{document}