\documentclass[12pt,a4paper,twocolumn]{article}
\usepackage{xparse}
\usepackage[utf8]{inputenc}
\usepackage{graphicx}
\usepackage{amsmath}
\usepackage{calligra}
\usepackage{asymptote}
\usepackage[version=4]{mhchem}
\usepackage{titlesec}
\setcounter{secnumdepth}{4}
\titleformat{\section}[block]{\sffamily\large\filcenter\bfseries}{\thesection.}{0.25cm}{\Large}
\titleformat{\subsection}[block]{\small\bfseries\sffamily}{\thesubsection.}{0.2cm}{\large}
\titleformat{\subsubsection}[block]{\small\sffamily}{\thesubsubsection}{0.18cm}{\large}


\usepackage{enumitem}

\newcommand{\scriptr}{\mathcalligra{r}\,}
\newcommand{\boldscriptr}{\pmb{\mathcalligra{r}}\,}
\usepackage{physics}
\usepackage{tcolorbox}
\usepackage{tikz}
\usepackage[a4paper]{geometry}
\usepackage{hyperref}
\author{\sffamily{\color{blue}{Advait Risbud}}}
\date{May 21'}
\title{\sffamily{MM 409: Colloids and Interface Science}}
\begin{document}
	\sffamily
	\maketitle
	\pagebreak
	\paragraph*{Specific Surface area of powders}
	\begin{equation}
	\boxed{	A_{sp}=\dfrac{3}{\rho R_s} }
	\end{equation}
Where $R_s$ is the radius of the equivalent sphere and $\rho$ is the density. 
\paragraph*{Laplace's Equation}
\begin{equation}
	\laplacian{\phi}=\pdv[2]{\phi}{x}+\pdv[2]{\phi}{y}+\pdv[2]{\phi}{z}=\dfrac{-\rho_e}{\epsilon_r\epsilon_0}
\end{equation}
The permittivity can change locally.

\paragraph*{Boltzmann distribution of ion concentration}
\begin{equation}
	\boxed{C_i=C_i^0 e^{\frac{-w_i}{k_BT}}}
\end{equation}
Where the $C_i$ denotes the concentration of the $i^{th}$ ion and $C_i^0$ is the concentration at the surface of the particle. $w_i$ denotes the work required to bring an ion in solution from infinity to a position closer to the surface. 

\paragraph*{Poisson-Boltzmann Equation in 1-D:}
\begin{align}
\pdv[2]{\phi}{x}&=\dfrac{C_0 Ze}{\epsilon_r \epsilon_0}\Bigg(e^{\dfrac{Ze\phi}{k_B T}}-e^{\dfrac{-Ze\phi}{k_B T}}\Bigg) \\
\pdv[2]{\phi}{x}&=\Bigg(\dfrac{2C_0 Ze^2}{\epsilon_0\epsilon_r k_B T}\Bigg)\phi \qq{for $\abs{e\phi}<<k_BT$}
\end{align}

\paragraph*{General Solution to P-B equation:}
\begin{align}
	\phi(x)&=C_1e^{\kappa x}+C_2e^{-\kappa x} \\
	\implies \phi(x)&=\phi_0 e^{-\kappa x}
\end{align}
where,
\begin{equation}
	\boxed{\kappa=\sqrt{\dfrac{2C_0 Ze^2}{\epsilon_r\epsilon_0 k_B T}}.}
\end{equation}
So the potential decays faster for electrolytes of higher charge($Z$) and concentration($C_0$) at the surface. 

\paragraph*{Debye Length}
\begin{equation}
	\lambda_D=\frac{1}{\kappa}
\end{equation}
Potential at the Debye length is,
\begin{equation*}
	\phi(\lambda_D)=\frac{\phi_0}{e}\approx \frac{\phi_0}{2.72}
\end{equation*}

\paragraph*{P-B equation in non-linearised case:}
$$\pdv[2]{\phi}{x}=\dfrac{C_0 Ze}{\epsilon_r \epsilon_0}\Bigg(e^{\dfrac{e\phi}{k_B T}}-e^{\dfrac{-e\phi}{k_B T}}\Bigg)$$
Let $y=\frac{Ze\phi}{k_B T}$,
\end{document}


















